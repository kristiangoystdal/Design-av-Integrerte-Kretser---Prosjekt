\appendix

\section{AIMSpice Code}
\label{appendix:aimspice}

\begin{lstlisting}[style=aimspiceStyle, caption=1-bit register in AIMSPICE, label=testcode]
1Bit Register

**************************************************************
* Including the file containing the NMOS and PMOS transistors
.include gpdk90nm_tt.cir
**************************************************************

**************************************************************
.PARAM W_VAL=1500n
.PARAM L_VAL=300n
**************************************************************

**************************************************************
* Subcircuit for a NOT gate 
.subckt NOT GND VDD A OUT
XMP1 OUT A VDD VDD PMOS1V W=W_VAL L=L_VAL 
XMN1 OUT A GND GND NMOS1V W=W_VAL L=L_VAL
.ends NOT
**************************************************************

**************************************************************
* Subcircuit for a NAND gate 
.subckt NAND GND VDD A B OUT
XMP1 VDD A OUT VDD PMOS1V W=W_VAL L=L_VAL 
XMP2 VDD B OUT VDD PMOS1V W=W_VAL L=L_VAL 
XMN1 OUT A C C NMOS1V W=W_VAL L=L_VAL 
XMN2 C B GND GND NMOS1V W=W_VAL L=L_VAL 
.ends NAND
**************************************************************

**************************************************************
* Subcircuit for an AND gate 
.subckt AND GND VDD A B OUT
XAND1 GND VDD A B C NAND
XNOT1 GND VDD C OUT NOT
.ends AND
**************************************************************

**************************************************************
* Subcircuit for a TRANSMISSION gate 
.subckt TRANS GND VDD IN EN_N EN_P OUT
XMN1 IN EN_N OUT GND NMOS1V W=W_VAL L=L_VAL
XMP1 IN EN_P OUT VDD PMOS1V W=W_VAL L=L_VAL 
.ends TRANS
**************************************************************

**************************************************************
* Subcircuit for a MUX for set and reset
.subckt MUX GND VDD S R DI Q DO
XNOT1 GND VDD S 1 NOT
XTRANS1 GND VDD DI S 1 2 TRANS
XTRANS2 GND VDD Q 1 S 2 TRANS
XNOT2 GND VDD R 3 NOT
XAND GND VDD 2 3 DO AND
.ends 
**************************************************************

**************************************************************
*Subcircuit for a D FLIP FLOP using TRANSMITION gates 
.subckt FLOP GND VDD CLK D Q 
XNOT1 GND VDD D 1 NOT
XNOT_CLK GND VDD CLK NOTCLK NOT
XTRANS1 GND VDD 1 NOTCLK CLK 2 TRANS
XNOT2 GND VDD 2 3 NOT
XNOT3 GND VDD 3 4 NOT 
XTRANS2 GND VDD 4 CLK NOTCLK 2 TRANS
XTRANS3 GND VDD 3 CLK NOTCLK 5 TRANS
XNOT4 GND VDD 5 6 NOT
XNOT5 GND VDD 6 7 NOT
XTRANS4 GND VDD 7 NOTCLK CLK 5 TRANS
XNOT6 GND VDD 6 Q NOT
.ends
**************************************************************

**************************************************************
* Subcircuit for 1bit register with set and reset
.subckt REGISTER GND VDD CLK S R D Q 
XMUX1 GND VDD S R D Q 1 MUX
XFLOP GND VDD CLK 1 Q FLOP
.ends
**************************************************************


**************************************************************
.PARAM RISE_TIME=0.5n 
.PARAM FALL_TIME=0.5n 
.PARAM CLK_PERIOD=20n 
.PARAM CLK_HIGH=10n 
.PARAM V_DD=0.85
.PARAM C_VDD=0.85


*Setting VDD = 0.85, the D, CLK, S and R as different square waves
VDD 1 0 V_DD
VD D 0 PULSE(0 C_VDD 25n RISE_TIME FALL_TIME 20ns 40ns)
VCLK CLK 0 PULSE(0 C_VDD 0 RISE_TIME FALL_TIME CLK_HIGH CLK_PERIOD)
VS S 0 PULSE(0 C_VDD 20n RISE_TIME FALL_TIME 60n 120n)
VR R 0 PULSE(0 C_VDD 145n RISE_TIME FALL_TIME 50n 100n)
**************************************************************

**************************************************************
*Defining the register
XREG 0 1 CLK S R D Q REGISTER
**************************************************************

**************************************************************
*Plotting Q, D, CLK, S and R for the register
.plot v(Q)
.plot v(D)
.plot v(CLK)
.plot v(S) 
.plot v(R)
**************************************************************
\end{lstlisting}


\section{Verilog Code}
\label{appendix:Verilog-code}

\begin{lstlisting}[style=verilogStyle, caption=Half Adder in Verilog, label=verilog_halfadder]
module Half_Adder ( A_0 ,B_0 ,S_0 ,C_0 );

input A_0 ;
wire A_0 ;
input B_0 ;
wire B_0 ;
output S_0 ;
wire S_0 ;
output C_0 ;
wire C_0 ;

and(C_0, A_0, B_0);
xor(S_0, A_0, B_0);

endmodule
\end{lstlisting}


\begin{lstlisting}[style=verilogStyle, caption=Full Adder in Verilog, label=verilog_fulladder]
module Full_Adder ( C_I ,S_O ,A_I ,C_O ,B_I );

input C_I ;
wire C_I ;
output S_O ;
wire S_O ;
input A_I ;
wire A_I ;
output C_O ;
wire C_O ;
input B_I ;
wire B_I ;

wire w1,w2,w3;
xor(w1, A_I, B_I);
and(w2, A_I, B_I); 
and(w3, w1, C_I);
xor(S_O, w1, C_I);
or(C_O, w2, w3); 

endmodule
\end{lstlisting}


\begin{lstlisting}[style=verilogStyle, caption=8-bit adder in Verilog, label=verilog_eightbitadder]
`include "Half_Adder.v"
`include "Full_Adder.v"

module eightbit_adder (
    input [7:0] A,
    input [7:0] B,
    output [7:0] S
);

wire [7:0] carry;
wire [7:0] carry_out;

Half_Adder ha_instance_0 (
    .A_0(A[0]),
    .B_0(B[0]),
    .S_0(S[0]),
    .C_0(carry[0])
);

Full_Adder FA_1 (
    .A_I(A[1]),
    .B_I(B[1]),
    .C_I(carry[0]),
    .S_O(S[1]),
    .C_O(carry_out[1])
);

Full_Adder FA_2 (
    .A_I(A[2]),
    .B_I(B[2]),
    .C_I(carry_out[1]),
    .S_O(S[2]),
    .C_O(carry_out[2])
);

Full_Adder FA_3 (
    .A_I(A[3]),
    .B_I(B[3]),
    .C_I(carry_out[2]),
    .S_O(S[3]),
    .C_O(carry_out[3])
);

Full_Adder FA_4 (
    .A_I(A[4]),
    .B_I(B[4]),
    .C_I(carry_out[3]),
    .S_O(S[4]),
    .C_O(carry_out[4])
);

Full_Adder FA_5 (
    .A_I(A[5]),
    .B_I(B[5]),
    .C_I(carry_out[4]),
    .S_O(S[5]),
    .C_O(carry_out[5])
);

Full_Adder FA_6 (
    .A_I(A[6]),
    .B_I(B[6]),
    .C_I(carry_out[5]),
    .S_O(S[6]),
    .C_O(carry_out[6])
);

Full_Adder FA_7 (
    .A_I(A[7]),
    .B_I(B[7]),
    .C_I(carry_out[6]),
    .S_O(S[7]),
    .C_O(carry_out[7])
);

endmodule
\end{lstlisting}


\begin{lstlisting}[style=verilogStyle, caption=D Flip-Flop in Verilog, label=verilog_dflipflop]
module D_Flip_Flop ( input D_flip , output Q_flip , input clk_flip );
	
	wire clk_bar_flip; 
	not(clk_bar_flip, clk_flip);
	not(w1_flip, D_flip);	  
	wire w2_flip;
	cmos(w2_flip, w1_flip, clk_bar_flip, clk_flip);	
	wire w3_flip;
	not(w3_flip, w2_flip); 
	wire w4_flip;
	not(w4_flip, w3_flip);
	cmos(w2_flip, w4_flip, clk_flip, clk_bar_flip);	 
	wire w5_flip;
	cmos(w5_flip, w3_flip, clk_flip, clk_bar_flip);
	wire w6_flip, w7_flip;
	not(w6_flip, w5_flip);
	not(w7_flip, w6_flip);
	cmos(w5_flip, w7_flip, clk_bar_flip, clk_flip);	 
	not(Q_flip, w6_flip);

endmodule
\end{lstlisting}


\begin{lstlisting}[style=verilogStyle, caption=Register control circuit in Verilog, label=verilog_regmux]
module Reg_mux (
    input S_mux,
    input R_mux,
    input D_I_mux,
    input Q_mux,
    output D_O_mux
);			  
wire Y_mux, sbar, rbar;	
not(sbar,S_mux);
not(rbar,R_mux); 

cmos(Y_mux,D_I_mux, S_mux, sbar);
cmos(Y_mux,Q_mux,sbar,S_mux);
and(D_O_mux, rbar, Y_mux);

endmodule
\end{lstlisting}


\begin{lstlisting}[style=verilogStyle, caption=One-bit register in Verilog, label=verilog_onebitregister]
`include "Reg_mux.v"
`include "D_Flip_Flop.v"

module onebit_register (input D, input S, input R, input CLK, output Q);

  wire D_O_mux; // Output from the control circuit

  Reg_mux mux_instance (
    .S_mux(S),
    .R_mux(R),
    .D_I_mux(D), 
    .Q_mux(Q), // Connect the Q output of the flip-flop to the input of the multiplexer  
    .D_O_mux(D_O_mux)
  );

  D_Flip_Flop flip_flop_instance (
    .D_flip(D_O_mux), // Connect the input of the flip-flop to the multiplexer's output
    .Q_flip(Q),
    .clk_flip(CLK)
  );

endmodule
\end{lstlisting}


\begin{lstlisting}[style=verilogStyle, caption=8-bit register, label=verilog_eightbitregister]
`include "onebit_register.v"

module eightbit_register (input s_eight , input r_eight ,input clk_eight, 
							input [7:0] d_eight ,output [7:0] q_eight );
	
	onebit_register onebit_instance0 (
	    .S(s_eight),
	    .R(r_eight),
		.CLK(clk_eight),
	    .D(d_eight[0]), // Connect the corresponding bit of d_eight to onebit_register
	    .Q(q_eight[0])
	); 
	onebit_register onebit_instance1 (
	    .S(s_eight),
	    .R(r_eight),
		.CLK(clk_eight),
	    .D(d_eight[1]), // Connect the corresponding bit of d_eight to onebit_register
	    .Q(q_eight[1])
	);
	onebit_register onebit_instance2 (
	    .S(s_eight),
	    .R(r_eight),
		.CLK(clk_eight),
	    .D(d_eight[2]), // Connect the corresponding bit of d_eight to onebit_register
	    .Q(q_eight[2])
	);		   
	onebit_register onebit_instance3 (
	    .S(s_eight),
	    .R(r_eight),
		.CLK(clk_eight),
	    .D(d_eight[3]), // Connect the corresponding bit of d_eight to onebit_register
	    .Q(q_eight[3])
	);
	onebit_register onebit_instance4 (
	    .S(s_eight),
	    .R(r_eight),
		.CLK(clk_eight),
	    .D(d_eight[4]), // Connect the corresponding bit of d_eight to onebit_register
	    .Q(q_eight[4])
	);
	onebit_register onebit_instance5 (
	    .S(s_eight),
	    .R(r_eight),
		.CLK(clk_eight),
	    .D(d_eight[5]), // Connect the corresponding bit of d_eight to onebit_register
	    .Q(q_eight[5])
	);
	onebit_register onebit_instance6 (
	    .S(s_eight),
	    .R(r_eight),
		.CLK(clk_eight),
	    .D(d_eight[6]), // Connect the corresponding bit of d_eight to onebit_register
	    .Q(q_eight[6])
	);
	onebit_register onebit_instance7 (
	    .S(s_eight),
	    .R(r_eight),
		.CLK(clk_eight),
	    .D(d_eight[7]), // Connect the corresponding bit of d_eight to onebit_register
	    .Q(q_eight[7])
	);

endmodule
\end{lstlisting}


\begin{lstlisting}[style=verilogStyle, caption=MAC in Verilog, label=verilog_MAC]
`include "twobit_multiplier.v"
`include "eightbit_adder.v"
`include "eightbit_register.v"

module MAC(
    input [1:0] A, 
    input [1:0] B, 
    input CLK, 
    input [1:0] CTRL, 
    output [7:0] Y
    );

    wire [1:0] A, B, CTRL;
    wire CLK;
    wire [7:0] Y;
    wire [3:0] PROD;
    wire [7:0] SUM;


    twobit_multiplier multiplier_instance(.A(A), .B(B), .O(PROD));
    eightbit_adder adder_instance(.A({4'b0000, PROD}), .B(Y), .S(SUM));
    eightbit_register register_instance(
        .s_eight(CTRL[0]), 
        .r_eight(CTRL[1]), 
        .clk_eight(CLK),
        .d_eight(SUM),
        .q_eight(Y));
    
endmodule
\end{lstlisting}


\begin{lstlisting}[style=verilogStyle, caption=FSM in Verilog, label=verilog_FSM]
`include "onebit_register.v"

module Mealy_FSM(
    input RunIN, 
    input ResetIN, 
    input CLK, 
    output RunOUT, 
    output ResetOUT
    );

    //Define wires

    wire RunIN, ResetIN, CLK, RunOUT, ResetOUT, Set, A, B;
    wire [1:0] C, N;

    //The FSM should always take in the next state

    assign Set = 1;

    //Define the registers

    onebit_register Reg0(.D(N[0]), .S(Set), .R(ResetIN), .CLK(CLK), .Q(C[0]));
    onebit_register Reg1(.D(N[1]), .S(Set), .R(ResetIN), .CLK(CLK), .Q(C[1]));

    //Throughput the reset-signal

    assign ResetOUT = ResetIN;

    //Define logic gates and MUX

    xor(N[0], C[0], RunIN);
    and(A, C[0], RunIN);
    xor(N[1], A, C[1]);
    nand(B, C[0], C[1]);
    and(RunOUT, B, RunIN);

endmodule
\end{lstlisting}


\begin{lstlisting}[style=verilogStyle, caption=Full MAC circuit in Verilog, label=verilog_fullmac]
`include "Mealy_FSM.v"
`include "MAC.v"

module MAC_FULL(
    input [1:0] I, 
    input [1:0] A, 
    input [1:0] B, 
    input CLK, 
    output [7:0] Y
    );

    wire CLK;
    wire [1:0] I, CTRL, A, B;
    wire [7:0] Y;

    Mealy_FSM fsm_instance(
        .RunIN(I[0]), 
        .ResetIN(I[1]), 
        .CLK(CLK), 
        .RunOUT(CTRL[0]), 
        .ResetOUT(CTRL[1])
    );
        
    MAC mac_instance(
        .A(A),
        .B(B),
        .CLK(CLK),
        .CTRL(CTRL),
        .Y(Y)
    );

endmodule
\end{lstlisting}

\section{Verilog Testbenches}
\label{sec:verilog_testbenches}

\begin{lstlisting}[style=verilogStyle, caption=Multiplier Testbench in Verilog, label=verilog_multiplierTB]
`timescale 1 ns / 1 ps
`include "twobit_multiplier.v"

module Multiplier_Testbench();
    reg [1:0] A, B;
    wire [3:0] O;
    integer errors;
    integer Acount, Bcount;

    twobit_multiplier mul_instance (.A(A), .B(B), .O(O));
    initial begin
        $dumpfile("Multiplier_Testbench.vcd");
        $dumpvars(0, Multiplier_Testbench);
        errors = 0;

        for(Acount = 0 ; Acount < 4 ; Acount++) begin
            for(Bcount = 0 ; Bcount < 4 ; Bcount++) begin
                A = Acount;
                B = Bcount;
                #1;
                if((A*B) != O) begin
                    $display("Error: ",A,"*",B,"=",O);
                    errors++;
                end else begin
                    $display(A,"*",B,"=",O);
                end
            end
        end
        if(errors==0)begin
            $display("No errors found");
        end else begin
            $display(errors," errors found. Reevaluate design.");
        end
        $finish;
    end

endmodule
\end{lstlisting}


\begin{lstlisting}[style=verilogStyle, caption=Adder Testbench in Verilog, label=verilog_adderTB]
`timescale 1 ns / 1 ps
`include "eightbit_adder.v"

module Adder_Testbench();
    reg [7:0] A, B;
    wire [7:0] S;
    integer errors;
    integer Acount, Bcount;

    eightbit_adder adder_instance(.A(A), .B(B), .S(S));

    initial begin
        $dumpfile("Adder_Testbench.vcd");
        $dumpvars(0, Adder_Testbench);
        errors = 0;
        for(Acount = 0 ; Acount < 256 ; Acount++) begin
            for(Bcount = 0 ; Bcount < 256 ; Bcount++) begin
                A = Acount;
                B = Bcount;
                #1;
                if((A+B) == S) begin
                    $display(A,"+",B,"=",S);
                end else begin
                    $display("Error: ", A, "+", B, "=",S);
                    errors++;
                end
            end
        end
        if(errors==0)begin
            $display("Congratulations! No errors found.");
        end else begin
            $display("WARNING! ",errors," errors found. Reevaluate circuit.");
        end
        $finish;
    end
endmodule
\end{lstlisting}


\begin{lstlisting}[style=verilogStyle, caption=Register Testbench in Verilog, label=verilog_registerTB]
`timescale 1 ns / 1 ps
`include "eightbit_register.v"

module Register_Testbench();
    parameter CLK_PERIOD = 10;
    reg s_eight,r_eight,clk_eight;
    reg [7:0] d_eight;
    wire [7:0] q_eight;
    integer errors;
    integer temp;

    eightbit_register register_instance(
        .s_eight(s_eight), 
        .r_eight(r_eight),
        .clk_eight(clk_eight),
        .d_eight(d_eight),
        .q_eight(q_eight)
    );

    always #((CLK_PERIOD / 2)) clk_eight = ~clk_eight;

    initial begin
        $dumpfile("Register_Testbench.vcd");
        $dumpvars(0, Register_Testbench);
        errors = 0;
        clk_eight = 0;

        //Setting a value in the register and checking output
        s_eight = 1;
        r_eight = 0;
        d_eight = 123;
        #CLK_PERIOD;

        //Testing controlsignals 00 for all inputs d
        s_eight = 0;
        r_eight = 0;
        for(temp = 0 ; temp < 256 ; temp++) begin
            d_eight = temp;
            #CLK_PERIOD;
            if(q_eight != 123) begin
                $display("Error: Value overwritten when S=0");
                errors++;
            end
        end

        //Testing controllsignals 01 and 11 for all inputs d
        s_eight = 0;
        r_eight = 1;
        for(temp = 0 ; temp < 256 ; temp++) begin
            d_eight = temp;
            #CLK_PERIOD;
            if(q_eight != 0) begin
                $display("Error: Output not zero when R=1");
                errors++;
            end
        end

        s_eight = 1;
        r_eight = 1;
        for(temp = 0 ; temp < 256 ; temp++) begin
            d_eight = temp;
            #CLK_PERIOD;
            if(q_eight != 0) begin
                $display("Error: Output not zero when R=1");
                errors++;
            end
        end

        //Testing all inputs d with control signals 10
        s_eight = 1;
        r_eight = 0;
        for(temp = 0 ; temp < 256 ; temp++) begin
            d_eight = temp;
            #CLK_PERIOD;
            if(q_eight != d_eight) begin
                $display("Error: Value not set when S=1");
                errors++;
            end
        end
        if(errors==0) begin
            $display("Congratulations! No errors found.");
        end else begin
            $display("WARNING! ",errors, " errors found. Reevaluate circuit.");
        end
        $finish;
    end
endmodule
\end{lstlisting}





\begin{lstlisting}[style=verilogStyle, caption=FSM Testbench in Verilog, label=verilog_FSMTB]
`timescale 1 ns / 1 ps
`include "Mealy_FSM.v"

module Testbench;

    reg CLK;
    reg [1:0] I;
    wire [1:0] O;

    always #5 CLK = ~CLK;

    Mealy_FSM uut(
        .RunIN(I[0]), 
        .ResetIN(I[1]), 
        .CLK(CLK), 
        .RunOUT(O[0]), 
        .ResetOUT(O[1]));

    initial begin

        $dumpfile("TB_dumpfile.vcd");
        $dumpvars(0, Testbench);

        CLK = 0;
        I = 3;
        #10;
        I = 1;
        #50;
        for(integer temp = 0 ; temp < 5 ; temp++)begin
            I = $random % 4;
            #40;
        end
        $finish;
    end

endmodule
\end{lstlisting}

\begin{lstlisting}[style=verilogStyle, caption=Full MAC Testbench in Verilog, label=verilog_FULLMAC_TB]
`timescale 1 ns / 1 ps
`include "MAC_FULL.v"

module TB;

    integer randomSeed = 5;
    reg [1:0] I, A, B;
    reg CLK;
    wire [7:0] Y;

    MAC_FULL uut(.*);

    always #5 CLK = ~CLK;
    always #10 A = $random(randomSeed) % 4;
    always #10 B = $random % 4;

    initial begin
        $dumpfile("TB_dumpfile.vcd");
        $dumpvars(0, TB);

        CLK = 0;
        I = 3;
        A = $random % 4;
        B = $random % 4;
        #10;
        I = 1;
        #100;
        I=0;
        #60;
        I = 3;
        #10;
        I = 1;
        #150;
        $finish;
    end

endmodule
\end{lstlisting}

\newpage
\section{AIMSpice Plots}
\label{appendix:aimspicePlots}

\subsection{Register Timing Plots}
\label{appendix:register_plots}

\begin{figure}[H]
    \begin{minipage}{0.5\textwidth}
        \centering
        \includegraphics[width=\textwidth]{Figures/Aimspice_Plots/TT_0.png}
        \caption{Plot of register for TT corner}
        \label{fig:TT0}
    \end{minipage}%
    \begin{minipage}{0.5\textwidth}
        \centering
        \includegraphics[width=\textwidth]{Figures/Aimspice_Plots/TT_27.png}
        \caption{Plot of register for TT corner}
        \label{fig:TT27}
    \end{minipage}
\end{figure}
\begin{figure}[H]
    \centering
    \begin{minipage}{0.5\textwidth}
        \centering
        \includegraphics[width=\textwidth]{Figures/Aimspice_Plots/TT_70.png}
        \caption{Plot of register for TT corner}
        \label{fig:TT70}
    \end{minipage}%
\end{figure}

\begin{figure}[H]
    \begin{minipage}{0.5\textwidth}
        \centering
        \includegraphics[width=\textwidth]{Figures/Aimspice_Plots/FS_0.png}
        \caption{Plot of register for FS corner}
        \label{fig:FS0}
    \end{minipage}%
    \begin{minipage}{0.5\textwidth}
        \centering
        \includegraphics[width=\textwidth]{Figures/Aimspice_Plots/FS_27.png}
        \caption{Plot of register for FS corner}
        \label{fig:FS27}
    \end{minipage}
\end{figure}
\begin{figure}[H]
    \centering
    \begin{minipage}{0.5\textwidth}
        \centering
        \includegraphics[width=\textwidth]{Figures/Aimspice_Plots/FS_70.png}
        \caption{Plot of register for FS corner}
        \label{figFS70}
    \end{minipage}%
\end{figure}

\begin{figure}[H]
    \begin{minipage}{0.5\textwidth}
        \centering
        \includegraphics[width=\textwidth]{Figures/Aimspice_Plots/SF_0.png}
        \caption{Plot of register for SF corner}
        \label{fig:SF0}
    \end{minipage}%
    \begin{minipage}{0.5\textwidth}
        \centering
        \includegraphics[width=\textwidth]{Figures/Aimspice_Plots/SF_27.png}
        \caption{Plot of register for SF corner}
        \label{fig:SF27}
    \end{minipage}
\end{figure}
\begin{figure}[H]
    \centering
    \begin{minipage}{0.5\textwidth}
        \centering
        \includegraphics[width=\textwidth]{Figures/Aimspice_Plots/SF_70.png}
        \caption{Plot of register for SF corner}
        \label{fig:SF70}
    \end{minipage}%
\end{figure}

\begin{figure}[H]
    \begin{minipage}{0.5\textwidth}
        \centering
        \includegraphics[width=\textwidth]{Figures/Aimspice_Plots/SS_0.png}
        \caption{Plot of register for SS corner}
        \label{fig:SS0}
    \end{minipage}%
    \begin{minipage}{0.5\textwidth}
        \centering
        \includegraphics[width=\textwidth]{Figures/Aimspice_Plots/SS_27.png}
        \caption{Plot of register for SS corner}
        \label{fig:SS27}
    \end{minipage}
\end{figure}
\begin{figure}[H]
    \centering
    \begin{minipage}{0.5\textwidth}
        \centering
        \includegraphics[width=\textwidth]{Figures/Aimspice_Plots/SS_70.png}
        \caption{Plot of register for SS corner}
        \label{fig:SS70}
    \end{minipage}%
\end{figure}

\begin{figure}[H]
    \begin{minipage}{0.5\textwidth}
        \centering
        \includegraphics[width=\textwidth]{Figures/Aimspice_Plots/FF_0.png}
        \caption{Plot of register for FF corner}
        \label{fig:FF0}
    \end{minipage}%
    \begin{minipage}{0.5\textwidth}
        \centering
        \includegraphics[width=\textwidth]{Figures/Aimspice_Plots/FF_27.png}
        \caption{Plot of register for FF corner}
        \label{fig:FF27}
    \end{minipage}
\end{figure}
\begin{figure}[H]
    \centering
    \begin{minipage}{0.5\textwidth}
        \centering
        \includegraphics[width=\textwidth]{Figures/Aimspice_Plots/FF_70.png}
        \caption{Plot of register for FF corner}
        \label{fig:FF70}
    \end{minipage}%
\end{figure}


\subsection{Register stability plots}
\label{appendix:register_stability}

\begin{figure}[H]
    \begin{minipage}{0.5\textwidth}
        \centering
        \includegraphics[width=\textwidth]{Figures/Aimspice_Plots/Temperatur0.png}
        \caption{\parbox{0.6\textwidth}{Plot of register stability for 0$^\circ$C for different corners.}}
        \label{fig:Temp0}
    \end{minipage}%
    \begin{minipage}{0.5\textwidth}
        \centering
        \includegraphics[width=\textwidth]{Figures/Aimspice_Plots/Temperatur27.png}
        \caption{\parbox{0.6\textwidth}{Plot of register stability for 27$^\circ$C for different corners.}}
        \label{fig:Temp27}
    \end{minipage}
\end{figure}
\begin{figure}[H]
    \centering
    \begin{minipage}{0.5\textwidth}
        \centering
        \includegraphics[width=\textwidth]{Figures/Aimspice_Plots/Temperatur70.png}
        \caption{\parbox{0.6\textwidth}{Plot of register stability for 70$^\circ$C for different corners.}}
        \label{fig:Temp70}
    \end{minipage}%
\end{figure}

\begin{figure}[H]
    \begin{minipage}{0.5\textwidth}
        \centering
        \includegraphics[width=\textwidth]{Figures/Aimspice_Plots/CornerSF.png}
        \caption{\parbox{0.6\textwidth}{Plot of register stability for different temperatures with SF corner.}}
        \label{fig:cornerSF}
    \end{minipage}%
    \begin{minipage}{0.5\textwidth}
        \centering
        \includegraphics[width=\textwidth]{Figures/Aimspice_Plots/CornerFS.png}
        \caption{\parbox{0.6\textwidth}{Plot of register stability for different temperatures with FS corner.}}
        \label{fig:cornerFS}
    \end{minipage}
\end{figure}
\begin{figure}[H]
    \begin{minipage}{0.5\textwidth}
        \centering
        \includegraphics[width=\textwidth]{Figures/Aimspice_Plots/CornerFF.png}
        \caption{\parbox{0.6\textwidth}{Plot of register stability for different temperatures with FF corner.}}
        \label{fig:cornerFF}
    \end{minipage}%
    \begin{minipage}{0.5\textwidth}
        \centering
        \includegraphics[width=\textwidth]{Figures/Aimspice_Plots/CornerSS.png}
        \caption{\parbox{0.6\textwidth}{Plot of register stability for different temperatures with SS corner.}}
        \label{fig:cornerSS}
    \end{minipage}
\end{figure}
\begin{figure}[H]
    \centering
    \begin{minipage}{0.5\textwidth}
        \centering
        \includegraphics[width=\textwidth]{Figures/Aimspice_Plots/CornerTT.png}
        \caption{\parbox{0.6\textwidth}{Plot of register stability for different temperatures with TT corner.}}
        \label{fig:cornerTT}
    \end{minipage}%
\end{figure}