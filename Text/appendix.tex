\appendix
\section{AIMSpice Code}
\label{appendix:aimspice}

\begin{lstlisting}[style=aimspiceStyle, caption=1-bit register in AIMSPICE, label=testcode]
**************************************************************
* Including the file containing the NMOS and PMOS transistors
.include gpdk90nm_tt.cir
**************************************************************

**************************************************************
.PARAM W_VAL=150n
.PARAM L_VAL=300n
**************************************************************

**************************************************************
* Subcircuit for a NOT gate 
.subckt NOT GND VDD A OUT
XMP1 OUT A VDD VDD PMOS1V W=W_VAL L=L_VAL 
XMN1 OUT A GND GND NMOS1V W=W_VAL L=L_VAL
.ends NOT
**************************************************************

**************************************************************
* Subcircuit for a NAND gate 
.subckt NAND GND VDD A B OUT
XMP1 VDD A OUT VDD PMOS1V W=W_VAL L=L_VAL 
XMP2 VDD B OUT VDD PMOS1V W=W_VAL L=L_VAL 
XMN1 OUT A C C NMOS1V W=W_VAL L=L_VAL 
XMN2 C B GND GND NMOS1V W=W_VAL L=L_VAL 
.ends NAND
**************************************************************

**************************************************************
* Subcircuit for an AND gate 
.subckt AND GND VDD A B OUT
XAND1 GND VDD A B C NAND
XNOT1 GND VDD C OUT NOT
.ends AND
**************************************************************

**************************************************************
* Subcircuit for a NOR gate 
.subckt NOR GND VDD A B OUT
XMP1 VDD A C VDD PMOS1V W=W_VAL L=L_VAL 
XMP2 C B OUT C PMOS1V W=W_VAL L=L_VAL 
XMN1 OUT B GND GND NMOS1V W=W_VAL L=L_VAL 
XMN2 OUT A GND GND NMOS1V W=W_VAL L=L_VAL 
.ends NOR
**************************************************************

**************************************************************
* Subcircuit for an OR gate 
.subckt OR GND VDD A B OUT
XGATENOR1 GND VDD A B C NOR
XNOT1 GND VDD C OUT NOT
.ends OR
**************************************************************

**************************************************************
* Subcircuit for a TRANSMISSION gate 
.subckt TRANS GND VDD IN EN_N EN_P OUT
XMN1 IN EN_N OUT GND NMOS1V W=W_VAL L=L_VAL
XMP1 IN EN_P OUT VDD PMOS1V W=W_VAL L=L_VAL 
.ends TRANS
**************************************************************

**************************************************************
* Subcircuit for a MUX for set and reset
.subckt MUX GND VDD S R DI Q DO
XNOT1 GND VDD S 1 NOT
XTRANS1 GND VDD DI S 1 2 TRANS
XTRANS2 GND VDD Q 1 S 2 TRANS
XNOT2 GND VDD R 3 NOT
XAND GND VDD 2 3 DO AND
.ends 
**************************************************************

**************************************************************
*Subcircuit for a D FLIP FLOP using TRANSMITION gates 
.subckt FLOP GND VDD CLK D Q 
XNOT1 GND VDD D 1 NOT
XNOT_CLK GND VDD CLK NOTCLK NOT
XTRANS1 GND VDD 1 NOTCLK CLK 2 TRANS
XNOT2 GND VDD 2 3 NOT
XNOT3 GND VDD 3 4 NOT 
XTRANS2 GND VDD 4 CLK NOTCLK 2 TRANS
XTRANS3 GND VDD 3 CLK NOTCLK 5 TRANS
XNOT4 GND VDD 5 6 NOT
XNOT5 GND VDD 6 7 NOT
XTRANS4 GND VDD 7 NOTCLK CLK 5 TRANS
XNOT6 GND VDD 6 Q NOT
.ends
**************************************************************

**************************************************************
* Subcircuit for 1bit register with set and reset
.subckt REGISTER GND VDD CLK S R D Q 
XMUX1 GND VDD S R D Q 1 MUX
XFLOP GND VDD CLK 1 Q FLOP
.ends
**************************************************************


**************************************************************
.PARAM RISE_TIME=0.1n 
.PARAM FALL_TIME=0.1n 
.PARAM CLK_PERIOD=20n 
.PARAM CLK_HIGH=10n 
.PARAM V_DD=0.85


*Setting VDD = 1, the CLK and D as two different pulses
VDD 1 0 V_DD
VD D 0 PULSE(0 V_DD 25n RISE_TIME FALL_TIME 20ns 40ns)
VCLK CLK 0 PULSE(0 V_DD 0 RISE_TIME FALL_TIME CLK_HIGH CLK_PERIOD)
VS S 0 PULSE(0 V_DD 35n RISE_TIME FALL_TIME 60n 120n)
VR R 0 PULSE(0 V_DD 145n RISE_TIME FALL_TIME 50n 100n)
**************************************************************

**************************************************************
*Defining the register
XREG 0 1 CLK S R D Q REGISTER
**************************************************************

**************************************************************
*Plotting D, CLK and Q for the register
.plot v(Q)
.plot v(D)
.plot v(CLK)
.plot v(S) 
.plot v(R)
**************************************************************

\end{lstlisting}


\section{Verilog Code}
Include all your verilog code here. You might want to include snippets of it earlier in the text as well, but the entire code should at least be included here.
\\ \\
Remember to format and comment your code for increased readability.

\begin{lstlisting}[style=verilogStyle, caption=D Flip-Flop in Verilog, label=verilog_dflipflop]
// Your Verilog code here
module my_module(
    input A,
    output B
);
    // Verilog code...
endmodule
...
\end{lstlisting}


\section{Optional (rename based on what you put here)}
Sometimes one might end up running a lot of simulations, and then find out that not all of them were relevant enough to present in the actual report. Additional figures and results can then be included here (with a brief explanation so that the reader knows what they are looking at). This section of the appendix is optional, and might not be relevant for your group. 

Anything added here will not affect the grade directly, but might contribute to the overall impression of the work you have done (which is part of what we grade). Your grade will NOT be affected negatively if only the AIMSpice and Verilog code is in the appendix.