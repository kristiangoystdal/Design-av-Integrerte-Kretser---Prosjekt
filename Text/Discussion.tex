\section{Discussion}
\label{sec: Discussion}


\subsection{AIMSpice}

As shown in \autoref{fig:result_TT27}, the different combinations of the inputs signals gives a Q-value based on \autoref{tab:registerFunc}. If the Set-value is high, Q only updates to the D-value when the CLK changes from low to high. And if the Reset-value is high, the value Q is overruled to low, even if Set and D is high. 

When looking at the different corners and temperatures, the difference of the functionality is quite similar. One can see in the case, \autoref{fig:TT_diffTemps}, with the TT corner at the different temperatures, that the register is just a tiny fraction more stable with a at higher temperatures.

For the case where we look at the different corners at the same temperature, \autoref{fig:27_diffCorners}, it is shown that the TT, FF and SF corners have the most ripples while the SS and FS corner have the least ripple.

% Discuss your results. You presented your results in the last section, so you do not need to repeat everything here. Just focus on the most interesting results and discuss these.

% Was there anything unexpected or weird about your results? What are possible explanations for these observations? It is good to draw on theory. This section and the theory section are the sections where it is natural to have the most references to literature (papers, the book, other).

% The discussion section must include a discussion of your results:
% \begin{itemize}
%     \item If the circuit did not work properly, \textit{why} might that be? 
%     \item Did some of your results turn out differently than you expected. \textit{Why}? You do not have to come to a definite answer, but should try to discuss at least one possible explanation.
%     \item It could be interesting to discuss the choices you made to reduce static power consumption, and potential improvements to these.
% \end{itemize}