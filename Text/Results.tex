\section{Results}
\label{sec:results}

\subsection{AimSpice}

To test the 1-bit register made by in AimSpice, we have to set some square waves for the clock signal, the data input, and the set and reset signals.

As shown in appendix \ref{appendix:aimspice}, line 95-98, we use the PULSE function in AimSpice to create square waves for the different inputs. These inputs will together determine what the output Q is set to. 

The register could have different effect and operations due to the corner of the transistor and the temperature. There are five corners the transistors could be; TT(typical-typical), SS(slow-slow), FF(fast-fast), SF(slow-fast) and FS(fast-slow). For all of the corners they have been tested for three temperatures, 0$^\circ C$, 27$^\circ C$ and 70$^\circ C$. All the different plots for the different cases are shown in the appendix \ref{appendix:aimspicePlots}. 

To validate the functionality of the register, one can examine the plot of the TT and FF corner at 27$^\circ C$. 

The register should have the functionality as shown in the table \ref{tab:registerFunc}. Where the value of Q gets set to D if set is low and it keeps its previous value if set is high. If reset is high, the value Q is set to low. 

\begin{table}[H]
\label{tab:registerFunc}
\centering
\caption{Functionality of 1bit-register}
\begin{tabular}{|l|l|l|}
\cline{1-3}
R & S & Q  \\ \cline{1-3}
0 & 0 & Q  \\ \cline{1-3}
0 & 1 & D  \\ \cline{1-3}
1 & 0 & 0  \\ \cline{1-3}
1 & 1 & 0  \\ \cline{1-3}
\end{tabular}
\end{table}

\begin{figure}[H]
    \centering
    \includegraphics[width=\textwidth]{Figures/Aimspice_Plots/TT_27.png}
    \caption{Plot of register for TT corner}
    \label{fig:result_TT27}
\end{figure}

As shown in figure \ref{fig:result_TT27} and figure \ref{fig:result_FF27}, the different combinations of the inputs signals. If set is high, Q only updates to the D-value when the CLK changes from low to high. And if the reset is high, the value Q is overruled to low, even if set and D is high.

When looking at the different corners and temperatures, the difference of the functionality is quite similar. One can see in this case with the TT and FF corners at 27$^\circ C$, that the register is just a tiny fraction more stable with a FF corner rather than a TT corner.

\begin{figure}[H]
    \centering
    \includegraphics[width=\textwidth]{Figures/Aimspice_Plots/FF_27.png}
    \caption{Plot of register for FF corner}
    \label{fig:result_FF27}
\end{figure}

\subsubsection{Static Power Consumption}

As explained in section \ref{subsec:low_power}, a low voltage on $V_{DD}$ is optimal. We have therefore chosen a value for $V_{DD} = 0.85V$, as a lower $V_{DD}$ gives a lower power consumption. By using the function operating point in AimSpice, the leakage current in the $V_{DD}$ node can be found.

Table \ref{tab:leakage} shows the different leakage current for the TT and FF corners with all three temperatures. To calculate the static power consumption, we use the formula \ref{eq:power} for all the values and get table \ref{tab:power}.

\begin{table}[H]
\centering
\caption{Leakage Current}
\label{tab:leakage}
\resizebox{0.4\columnwidth}{!}{%
    \begin{tabular}{lll}
    \cline{2-3}
    \multicolumn{1}{l|}{} &
      \multicolumn{1}{l|}{TT} &
      \multicolumn{1}{l|}{FF} \\ \hline
    \multicolumn{1}{|l|}{0 $^\circ$C} &
      \multicolumn{1}{l|}{1.0931 nA} &
      \multicolumn{1}{l|}{3.1854 nA} \\ \hline
    \multicolumn{1}{|l|}{27 $^\circ$C} &
      \multicolumn{1}{l|}{3.0325 nA} &
      \multicolumn{1}{l|}{8.5711 nA} \\ \hline
    \multicolumn{1}{|l|}{70 $^\circ$C} &
      \multicolumn{1}{l|}{6.9843 nA} &
      \multicolumn{1}{l|}{13.513 nA} \\ \hline
     &  &  
    \end{tabular}%
}
\end{table}

\begin{table}[H]
\centering
\caption{Static Power Consumption}
\label{tab:power}
\resizebox{0.4\columnwidth}{!}{%
\begin{tabular}{lll}
\cline{2-3}
\multicolumn{1}{l|}{} &
  \multicolumn{1}{l|}{TT} &
  \multicolumn{1}{l|}{FF} \\ \hline
\multicolumn{1}{|l|}{0 $^\circ$C} &
  \multicolumn{1}{l|}{0.92914 nW} &
  \multicolumn{1}{l|}{2.7076 nW} \\ \hline
\multicolumn{1}{|l|}{27 $^\circ$C} &
  \multicolumn{1}{l|}{2.5776 nW} &
  \multicolumn{1}{l|}{7.2854 nW} \\ \hline
\multicolumn{1}{|l|}{70 $^\circ$C} &
  \multicolumn{1}{l|}{5.9367 nW} &
  \multicolumn{1}{l|}{11.486 nW} \\ \hline
 &  &  
\end{tabular}
}
\end{table}

\subsection{Verilog}

In this subsection results of Verilog-simulation will be presented. All Verilog-code is given in \autoref{appendix:Verilog-code}.

\autoref{fig:fsm_simulation} shows the FSM simulated for randomized inputs $I_1$ and $I_0$.

\begin{figure}[H]
    \centering
    \includegraphics[width=\textwidth]{Figures/Test of FSM.png}
    \caption{Timing diagram of FSM simulated in Verilog}
    \label{fig:fsm_simulation}
\end{figure}

From the figure we see that the memory transitions to the expected states based on the inputs. The FSM begins in state ''Run-1'' and transitions to state ''Reset'' as the reset input $I_1$ is set high. The next state is ''Run-1'', then ''Run-2''. The Run-bit $I_0$ is then set low, resulting in a two-period pause in state ''Pause-2''. Further the FSM transitions to ''Run-3'', ''Pause'' and ''Run-1''.

This simulation was run and verified to follow the wanted behaviour as given by table x.

\autoref{fig:eightbitadder_sim} shows a simulation of the 8-bit adder with random 8-bit inputs A and B. Note that all values in the timing diagram are hexadecimal and that any overflow-bits are handled outside our system. The code used for this simulation is given in \autoref{verilog_eightbitadder}.

\begin{figure}[H]
    \centering
    \includegraphics[width=\textwidth]{Figures/Test of eightbitadder.png}
    \caption{Timing diagram of 8-bit adder simulated in Verilog}
    \label{fig:eightbitadder_sim}
\end{figure}

\autoref{fig:dflipflop_sim} shows a simulation of a single D Flip-Flop realized in Verilog.

\begin{figure}[H]
    \centering
    \includegraphics[width=\textwidth]{Figures/Test of Dflipflop.png}
    \caption{Timing diagram of D Flip-Flop simulated in Verilog}
    \label{fig:dflipflop_sim}
\end{figure}

\autoref{fig:8bitregister_sim} shows a simulation of a 8-bit register realized in Verilog.

\begin{figure}[H]
    \centering
    \includegraphics[width=\textwidth]{Figures/VerilogPlot_8bitreg.png}
    \caption{Timing diagram of an 8-bit register simulated in Verilog}
    \label{fig:8bitregister_sim}
\end{figure}

\autoref{fig:multiplier_sim} shows a simulation of the 2-bit multiplier realized in Verilog. Inputs A and B are stimulated with arbitrary two bit values to observe the possible outputs. The multiplier has a 4-bit output as the largest possible output result, 9, needs four bits to be represented in binary.

\begin{figure}[H]
    \centering
    \includegraphics[width=\textwidth]{Figures/Test of multiplier.png}
    \caption{Timing diagram of multiplier simulated in Verilog}
    \label{fig:multiplier_sim}
\end{figure}



% Present the results of your simulations in this section. Use tables and graphs or other figures to illustrate your results. Remember: The table caption goes above the table, the figure caption goes below the figure.

% The results section must include:
% \begin{itemize}
%     \item Figures and/or tables that show the results of your simulation.
%     \item Text that describe what we see in the simulation results (e.g. as expected we can see that XYZ which means the circuit functions as intended).
%     \item NB! The result section is a \textit{what?}-section. \textit{What} where the results? \textit{What} do the figures/results mean? Any \textit{why}-questions you might want to write about and try to answer typically belong in the discussion section.
% \end{itemize}